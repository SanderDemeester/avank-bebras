\begin{document}
\section{Instalation guide}
\subsection{Requirements}
\begin{itemize}
\item sbt build tool
\item java 1.7 (at least)
\item postgreql 9.1 (at least)
\end{itemize}

\subsection{Server configuration - postgresql setup}
\begin{itemize}
\item Install postgres
\item[First] By default our application uses the default user "postgres". Please change the password for this user to "sander". This is the default for database access but can be changed in
\item[second] create a database "avank" using the psql tool.
\item[second] psql -U postgres.
\item[second] create database avank;
\item[second] \\i <path to createTablesComplete>
\item[second] \\i <path to populateDatabase.sql>
\item[First] *Optional: Optional: if you cannot login using the "psql" command. You should change the authentication methode in "pg_hba.conf" to "md5". 
"conf/application.conf" using the key "db.default.password".
\end{itemize}
\subsubsection{Building the application}
\begin{itemize}
\item[First] Install the scala build tool "sbt".
\item[Second] If you are using a debian based package manager, you can use the apt.typefase repository.
\item[Second] wget http://apt.typesafe.com/repo-deb-build-0002.deb
\item[Second] sudo dpkg -i repo-deb-build-0002.deb
\item[Second] sudo apt-get update
\item[Second] sudo apt-get install sbt
\item[First] Download latest source from github: https://github.ugent.be/fvdrjeug/avank-bebras
\item[First] checkout branch "stable"
\item[First] running the application should be as simple as typing "sbt compile stage" and running "target/start" in the application folder.

\item[First] If you would like to run the application as a standalone app (this is recommended) please use the following instructions.
\item[First] Download the zip file "name application"
\item[First] You can (if you want) use a different "application.conf" file. This will allow you to use different default values for yor application (this is recommended).
       You will need to build your own production zip file. The steps on how to do this can be found at the play website: http://www.playframework.com/documentation/2.0/ProductionDist
\item[First] unzip the zip file "unzip avank-1.0-SNAPSHOT.zip"
\item[First] execute "chmod u+x avank-1.0-SNAPSHOT/start"
\item[First] To start the application, execute "./avank-1.0-SNAPSHOT/start".
\item[First] The default port is "9000", you can change this with command line argument "-Dhttp.port=<port number>", so the full command would then be: "./avank-1.0-SNAPSHOT/start -Dhttp.port=80".
      
\item[First] Optional, it is possible to install the complete play! Framework. This should be play version 2.0.4 and can be found: http://downloads.typesafe.com/releases/play-2.0.4.zip
\end{itemize}
\end{document}