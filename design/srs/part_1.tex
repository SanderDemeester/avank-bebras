
\section*{1 Overview}

The web application supports three different kinds of contests:

\begin{itemize}
    \item The official competition, which can only be taken during the official
        Bebras contest week.
    \item Restricted contests, which can only be taken by registered pupils.
        Such competitions can be run under supervision of a teacher. These
        contests can serve as a tryout for the official competition, or can be
        used in the context of a computer science lesson.
    \item Unrestricted quizzes, which anyone can do, without the need for
        registration.
\end{itemize}

Each contest consists of a series of questions which must be answered online,
within a restricted time period. Because this a Belgian product, all questions
are provided in the three official languages, and also the user interface is
multilingual. A language is chosen when the user first enters the application.
(A language preference is recorded when the user registers.)

Question sets are different for pupils of different age, although question sets
may overlap, and sometimes the same question is given less or more weight
according to the age of the contestant.  Questions can even be shared among
different competitions.

Most questions are in a multiple choice format (with four possible answers)
although some ques- tion require a simple answer (a word, a number, ...) whose
correctness can be verified (by the application) by matching it against a
regular expression.

A question is presented as an HTML page, possibly with some additional
JavaScript. These pages are prepared outside of the application and are
uploaded separately.

The application allows for a large number of simultaneous users (for example,
in 2011 there were more than 11000 participants in the Netherlands). For that
reason it must be possible to store the questions and answer pages on a
(cluster of) web server(s) that is separate from the application server.

Pupils can only register through their teachers. Pupils obtain a `Bebras ID'
which remains the same over the years, even if they switch classes or schools.
Teachers register by e-mail.

