\section*{Questions, question sets and contests}

	A \emph{question} is essentially a folder of HTML files, JavaScript files, images, ....
	By default the index page of that folder is displayed by the application. In fact,
	every question consists of several folders, one for each supported language. The same
	folder also has a feedback page per language, which contains (and explains) the
	correct answer for that question and optionally illustrates what the relation is
	between the question and the skills needed to be a good computer scientist.
	Possible examples for those questions are available on the German website.
	These folders are stored outside the application on a separate server (a standard
	HTTP server). The question pages themselves do not contain the code needed to submit
	the answers: question authors need not know about the technical details of the
	application. The only thing the question authors need to be aware of, is the folder
	structure, the 'configuration'-file and the multiple languages for that question. \\
	\\
	The folder names of these questions must have a unique identifier on the current
	server. There can be multiple servers that provide questions so in fact questions
	are identified by their identifier and the server they are on. Question servers can be
	added to the application so that their questions can be used in question sets. The
	question folders are not managed from within the application, only the links to them.
	\\
	\\
	The configuration file will be xml-based and will contain options like:
	\begin{itemize}
		\item The available languages of the question.
		\item The start page of the question for each available language.
		(example index\_nl.html and index\_fr.html)
  		\item Title of the question for all available languages.
  		\item Question type:
  		\begin{itemize}
  			\item Multiple choice: Amount of questions with one marked as correct.
  			This will an xml-based list and will be displayed in the application as a
  			lexicographical ordered list identified by the A, B, C, ...
  			The maximum amount of questions is obviously limited to 24.
  			The question tag will contain multiple translated question depending on the
  			available languages for this question.
  			\item Short answer: Regular expression of the expected answer for each
  			available language. These can be words, sentences, numbers, ... As long as
  			they are accepted by the given expression.
  		\end{itemize}
	\end{itemize}
	Only the answer input will be generated by the application, but the question and
	optional extra information have to be provided in the html-file by the question author.
	The question author is free to add any other content he likes, as long as the
	configuration file is valid. \\
	\\
	When questions are presented to the users, they will also see the two types of
	questions. Multiple choice questions must by answered by pressing buttons labeled as
	A, B, C, ... as configured by the question author. The short answers have to be
	entered in a normal textfield which then is checked with the provided regular
	expression given by the question author.\\
	\\
	Question sets can be created by the organizers or teachers by selecting a list of questions.
	For each question in this list, a difficulty has to be selected. The final score of a
	competition is calculated as stated in following table.
	\begin{tabular}{ c c c c }
	  & Easy & Medium & Hard \\
	  Correct & +6 & +9 & +12 \\
	  Wrong & -2 & -3 & -4 \\
	  No answer & 0 & 0 & 0 \\
	\end{tabular}
	The question sets must also store a time limit and optionally some remarks about the
	set that are only visible by the organizers who will use them. \\
	\\
	These question sets will be used to create competitions which contain multiple
	question sets, each with a different level. There can be one or more levels, each
	one connected to a language level. Competitions will only be available to users if
	they are assigned to them and if the competitions contain a level that corresponds to
	the level the user has. Competitions can only be created by organizers and teachers where they have
	to select the available languages and they also have to provide a title for each
	language. Upon creation all the questions will be checked and the creator will be
	notified if questions are not available in all of the languages chosen for the
	competition.\\
	\\
	The same question may belong to different question sets with different marks and
	difficulty indication. Typically, the same question will occur two or three times in
	the same competition, but at different levels. It is also allowed for the same
	question to occur in different competitions. Complete question sets are however never
	shared, so these sets are directly linked to the competitions. Typical titles for
	competitions are ``Bebras November 2011'', ``Best of Bebras 2007-2010'', ...
	
