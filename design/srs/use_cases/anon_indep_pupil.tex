\begin{section}{Anonymous/Not logged in}


    \begin{subsection}{Register.} The landing page for a not logged in user consists of several elements. One of these is a Register button. Clicking this brings the user to a registration page. This page has several text fields (name, gender, e-mailadress (optional), date of birth, preferred language, password, confirm password) and a Confirm button. If the form is filled out incorrectly (2 passwords not equal, password not according to specifications, required field empty,...) and the user clicks Confirm, he is informed of this and given the option to fix it. If it is filled out correctly and the user clicks Confirm, the registration is succesfull and the user is brought to the landing page for an Independent user.\end{subsection}
	\begin{subsection}{Do unrestricted quizzes.} The landing page for a not logged in user also has a link to the unrestricted quizzes. Clicking this brings them to a list of all the unrestricted quizzes. Each quiz has a Start button. Clicking this brings them to the startpage of the quiz.\end{subsection}
    \begin{subsection}{Change preferred language (not persistent)}The landing page also has a language selection item. Selecting the preferred language changes the language of the application for the session.\end{subsection}
    \begin{subsection}{Log in}On the landing page there are also a number of fields for logging in (Bebras ID and password) and a button. If the ID-password combo is a valid one, the user is brought to the landing page for his User Role. If it isn't he is informed of this so he can fix the errors. \end{subsection}


\end{section}
\begin{section}{Independent}

    \begin{subsection}{Change password}
		\label{indep-pass}
		On the landing page of an Independent user there is a link to a Settings page. This page has, amongst others, several fields for the purpose of changing the password (current password, new password, repeat new password). If these are filled out correctly (current password is correct, new password is repeated correctly, new password is according to specifications), the password is changed. If they aren't, then the used is informed of this.
    \end{subsection}
    \begin{subsection}{Change e-mail address}
    	\label{indep-email}
    \end{subsection}
    \begin{subsection}{Change preferred language}
    	\label{indep-lang}
    \end{subsection}
    \begin{subsection}{Do unrestricted quizzes}
    	\label{indep-unrest}
    \end{subsection}
    \begin{subsection}{View results and statistics from quizzes/contests he previously participated in}
    	\label{indep-resstat}
    \end{subsection}
    \begin{subsection}{View the classes he has been associated with}
    	\label{indep-class}
    \end{subsection}
    \begin{subsection}{Logout}
    	\label{indep-logout}
    \end{subsection}
    
    \end{section}
    

\begin{section}{Pupil}

    \begin{subsection}{Change password}Equal to \ref{indep-pass}.\end{subsection}
    \begin{subsection}{Change e-mail address}Equal to \ref{indep-email}.\end{subsection}
    \begin{subsection}{Change preferred language}Equal to \ref{indep-lang}.\end{subsection}
    \begin{subsection}{Do unrestricted quizzes}Equal to \ref{indep-unrest}.\end{subsection}
    \begin{subsection}{Participate in contests that have been opened for his class by the teacher (both restricted as official)}TODO\end{subsection}
    \begin{subsection}{View results and statistics from quizzes/contests he previously participated in}Equal to \ref{indep-resstat}\end{subsection}
    \begin{subsection}{View the classes he has been associated with and his current class}Similar to \ref{indep-class}. The only difference is that he can also see which class he is currently registered in.\end{subsection}
	\begin{subsection}{Logout}Equal to \ref{indep-logout}\end{subsection}

\end{section}