

\begin{section}{Organizers}
	When the organizer is succesfully signed in, he will be redirected to his dashboard.
	The organizer also has the default user functionalities that have been previously
	mentioned. But he will also be some extra links presented that are exclusively
	available to organizers.
	\begin{subsection}{Manage Pupils}
		\label{sec:organizer_manage_pupils}
		Upon entering this page, there will be a table visible with all the (basic) users
		registered in the bebras application. Columns include but are not limited to:
		\begin{itemize}
			\item Bebras-ID
			\item Name
			\item Email Adress
			\item Class, [School]
			\item Gender
			\item Age
			\item Actions
		\end{itemize}
		The actions column contains multiple buttons that can perform different actions
		for each user, these actions are the following:
		\begin{itemize}
			\item Delete
			\item Edit
			\item Block
			\item Mimic
		\end{itemize}
		The first row of the column will contain only input fields that will serve as
		search input. For example the organizer can search a user by name and class by
		entering (part) of the pupil's name and the id of the class. The table will
		automatically update when the search button is pressed and will only contain
		rows that satisfy the input query.\\
		\\
		The table won't show all rows at al time, but it will use a paging system that
		by default shows only twenty rows at a time.\\
		\\
		There will also be a prominent Add button visible. After clicking it, the
		organizer will be redirected to a page identical to
		\nameref{sec:organizer_edit_pupils}, with the only difference that no fields
		are filled in. \\
		\\
		Each row also has a select box. When multiple rows are selected, the organizer is
		able to perform multi-actions, these include:
		\begin{itemize}
			\item Delete
			\item Edit ClassGroup
			\item Block
		\end{itemize}
		These multi-actions are useful for when larger groups of pupils require similar
		actions. Clicking a select-box in the column-name row of the table will select
		all the rows on the current page.
		The first action will simply remove the selected pupils, but the other actions
		redirect the organizer to edit pages identical to their single-edit equivalent.
		\begin{subsubsection}{Delete Pupils}
			When the organizer whishes to remove a pupil, he can click on the delete
			button in the row of that user. A confirmation dialog will be shown after
			clicking it, to make sure the organizer really wants to delete that user. \\
			The user will not really be deleted from the database, but only unlinked so
			that the statistics are in noway harmed.
		\end{subsubsection}
		\begin{subsubsection}{Edit Pupils}
			\label{sec:organizer_edit_pupils}
			The Edit button will redirect the user to a dedicated edit-screen where all
			the fields of a pupil that are available in the database are presented in a
			form and can be modified. Every field of the form is a regular text-field.
			Upon saving the form by clicking the button on the bottom of the form, the
			form will be checked and the organizer will be notified if there are errors
			in the form and where they are. For example a non-existing ClassGroup id.
			When there were no errors while editing, the organizer will be redirected to
			the main pupil-management page.
		\end{subsubsection}
		\begin{subsubsection}{Block Pupils}
			When pupils are misbehaving the organizer has an option to temporarily or
			permanently block a user from the application. When the organizer clicks on
			the Block button, he will be redirected to a new page where he will presented
			to a small form where he has to choice to select not-blocked, permanent or
			temporarily. When he chooses temporarily a new input box will become visible
			where he can choose the date for when the user should become active again.
			When the pupil was already blocked. When a user is already being blocked,
			the organizer is able to visit this screen and change the current
			blocking-mode.
		\end{subsubsection}
		\begin{subsubsection}{Mimic Pupils}
			Upon pressing this button, the organizer will be redirected to the dashboard
			of the selected pupil. He will be able to do anything that pupil can, and
			nothing else. Only when the disguised organizer clicks on the log-out button,
			he will be redirected to the page he left off, in this case the Pupil
			managment.
		\end{subsubsection}
		
	\end{subsection}
	
	\begin{subsection}{Manage Teachers}
		The Teacher management page is identical to the Pupil management page, with the
		only difference that there is also a phone-number field that is only available
		to the teacher.
	\end{subsection}
	
	\begin{subsection}{Manage ClassGroups}
		This page is identical to the ClassGroup managment page that is available to the
		teachers, but the organizers are able to see all the available classgroups that
		are present in the application. The organizer is also able to change the ownership
		of the class to another teacher in the Edit ClassGroup form.
	\end{subsection}
	
	\begin{subsection}{Manage Questions}
		This page contains two tables that are positioned next to each other. \\
		The first table contains all the servers that provide the hosting of questions.
		This is nothing more than a simple list of names of the server, base-url and
		optional additional path to the questions base folder. \\
		The second table contains all the questions and has the following columns:
		\begin{itemize}
			\item ID
			\item Server
			\item Official/Regular
			\item Active
		\end{itemize}
		The Server is field contains the name of a certain server specified in the first
		table and the ID is the name of a subfolder on that server that contains all the
		resource files necessary for that question.
		The last two fields are simple toggles that say whether the question is official
		or regular and whether the question is active or not. Non-active questions are
		not usable in the rest of the application, they can be useful when a question
		needs evaluation or is not yet fully finished. \\
		\\
		These table also have Add buttons at the top, and Edit and Delete buttons in each
		row that perform equivalent actions as in \nameref{sec:organizer_manage_pupils}.
		\\
		\\
		Each row in the questions table also have select boxes that are needed to perform
		the following multi-actions:
		\begin{itemize}
			\item Change server
			\item Set official question
			\item Set regular question
			\item Activate
			\item De-activate
		\end{itemize}
		The first multi-action will redirect the user to a simple form page where he can
		select a server for the selected questions. And the other actions are simple
		toggles that happen on this page.\\
		\\
		The questions table also has a first row with input fields that serves as
		search fields. The last two column are no textual inputs, but simple dropdown
		list with the the values None, Official and Regalar for the first one and the
		values None, Active and Inactive for the second one.
	\end{subsection}
	
	\begin{subsection}{Manage Question-sets}
		
	\end{subsection}
	
	\begin{subsection}{Manage Competitions}
		
	\end{subsection}
	
	\end{section}