
\begin{section}{General User}

    This section describes the shared functionality of all registered users,
    along with the registration, logging in and home page functionality.

    \begin{subsection}{Home Page}

        As this is the first page users see, it's all but obligated to start our
        list of use cases here. Figure \ref{img:index} displays a simple example
        home page.

        Starting at this home page, the user can complete several actions. These
        are described below.

        \begin{figure}[b]
            \centering
            \includegraphics[width=0.5\textwidth]{img/index.png}
            \caption{Simplified home page}
            \label{img:index}
        \end{figure}

        \begin{subsubsection}{Log in}
            On the landing page there are also a number of fields for logging
            in (Bebras ID and password) and a button. If the ID-password combo
            is a valid one, the user is brought to the landing page for his
            User Role. If it isn't he is informed of this so he can fix the
            errors.
        \end{subsubsection}

        \begin{subsubsection}{Register}
            The landing page for a not logged in user consists of several
            elements. One of these is a Register button. Clicking this brings
            the user to a registration page. This page has several text fields
            (name, gender, e-mailadress (optional), date of birth, preferred
            language, password, confirm password) and a Confirm button. If the
            form is filled out incorrectly (2 passwords not equal, password not
            according to specifications, required field empty,...) and the user
            clicks Confirm, he is informed of this and given the option to fix
            it. If it is filled out correctly and the user clicks Confirm, the
            registration is succesfull and the user is brought to the landing
            page for an Independent user.
        \end{subsubsection}

        \begin{subsubsection}{Forgot password}
        \end{subsubsection}

        \begin{subsubsection}{Go home}
            On the home page is a link to the actual www.Bebras.be site, where
            people can get more information on the organization.
        \end{subsubsection}

        \begin{subsubsection}{Participate in unrestricted contests}
            The landing page of an independent user has a link to the
            unrestricted quizzes page. This page has a list of all unrestricted
            quizzes he can do (except the ones he has already done). Each quiz
            has a start button which brings the user to the starting page of
            the quiz.
        \end{subsubsection}

        \begin{subsubsection}{Contact a human}
            This link leads to a page containing form for contacting organizers
            and administrators. Here, the user can select a topic, so mail can
            easily be organized for the readers. A reply-to address must be
            provided.
        \end{subsubsection}

        \begin{subsubsection}{Read Frequently Asked Questions}
            This link on the home page leads you to a page with some frequently
            asked questions. These are indexed and can be searched. The
            organizers can edit these questions and their answers.
        \end{subsubsection}

    \end{subsection}

    \begin{subsection}{Shared Functionality}

        This subsection describes the operations all (registered) users have in
        common. This includes functionality like logging out, changing personal
        information, etc...

        \begin{subsubsection}{Log out}
            Every page except for quiz and contests pages has a log out button.
            Pressing this button logs the user out of the system and brings him
            to the landing page for a not logged in user.
        \end{subsubsection}

        \begin{subsubsection}{Change personal information}
            When any registered user goes to this page using the link on his
            landing page, he can view and change (some of) his personal
            information. An example layout for the page would be Figure
            \ref{img:personal}.

            When the user clicks on any of the ``edit'' links, the corresponding
            data field will turn into a form element, and a simple save button
            needs to be clicked to safe the new value. This could look a bit
            like Figure \ref{img:personal_edit}.

            \begin{figure}
                \centering
                \begin{subfigure}{0.3\textwidth}
                    \includegraphics[width=0.7\textwidth]{img/personal_u.png}
                    \subcaption{As the page is loaded.}
                    \label{img:personal}
                \end{subfigure}

                \begin{subfigure}{0.3\textwidth}
                    \includegraphics[width=0.7\textwidth]{img/personal_e.png}
                    \subcaption{When editing the email-address.}
                    \label{img:personal_edit}
                \end{subfigure}
                \caption{Simplified layout of the personal page form.}
            \end{figure}
        \end{subsubsection}

        \begin{subsubsection}{View statistics}
        \end{subsubsection}

    \end{subsection}

\end{section}


