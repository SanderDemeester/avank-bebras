
\begin{section}{Teachers}

    This section provides an overview of all functionality a teacher is offered
    on our application.

    \begin{subsection}{Inheritance}

        The teacher can do many things also available for other users. These are
        listed below. More explanation can be found with the respective users
        they share these functionality with.

        \begin{subsubsection}{Independent}

            \begin{itemize}
                \item Change password           (\ref{indep-pass});
                \item Change email address      (\ref{indep-email});
                \item Change preferred language (\ref{indep-lang});
                \item Log out                   (\ref{indep-logout}).
            \end{itemize}

        \end{subsubsection}

    \end{subsection}

    \begin{subsection}{General functionality}

        The teacher will use a lot of specifications given by the organizer
        (like managing personal information, pupils, classes etc.). In general a
        teacher can do the following actions.

        \begin{subsubsection}{Change personal information}
            When the teacher logs in he will come to a personal web page. On
            this page, at first, he can see:

            \begin{itemize}
                \item Personal information
                \item School's information
                \item Login information
            \end{itemize}

            Beneath the personal information form there is placed an edit button
            which provides an edit functionality. The School's information form
            shows information about the schools to which the teacher is linked.
            Also there is some general information for each school like the
            website, name, email address and telephone number.

            Under the login information part there is an 'Edit password' button. 
            This button brings the teacher to a webpage where he can fill in his current password and new password. 
            After clicking 'Change', the current password is being checked before the changes take place.
        \end{subsubsection}
        \begin{subsubsection}{Register pupil/class}
            Information of this part can be found under section \nameref{sec:organizer}.\\
            Although it is import to mention that the registration and assigning is done by uploading a CSV file to the server, it's also important to note that there is only one problem with the use of a CSV file. When a pupil is registered and he	was already registered before (to another class), this pupil must be merged to the current ID in order to avoid
            redundancy.\\
            So a check for every pupil must be done while uploading a CSV. \\
            When it comes to registering a class, the teacher goes to a special web page called the classgroup web page. On this page
            the teacher can add a classgroup to an existing list of available classgroups within the application. There is a button
            'Add classgroup' which brings the teacher to a page where he can see a list with all the registrated pupils. Next to every
            pupil there is a green '+' button and '-' button (if the pupil is already in the classgroup list). These add/remove buttons
            are used in order to add/remove a pupil from the current classgroup which is being registered.\\
            \begin{figure}[!h]
                \centering
                \includegraphics[width=1\textwidth]{formal/img/add_remove_pupil_classgroup.png}
                \caption{How creating a classgroup is done}
                \label{create_classgroup}
            \end{figure}
            After all the pupils are in the list, the teacher can click 'Add classgroup' which adds this group to the existing list
            of classgroups. 
        \end{subsubsection}
        \begin{subsubsection}{Open and close (local) competition}
            This is also very similar as in the section about organizers. The teacher will use the organizer's competition page.
            On this page the teacher can simply add classes which he has registered as described in section Register pupil/ class.
            A class must be added in order to participate in the official Bebras competition. After adding a class the teacher can
            close the competition.\\
        \end{subsubsection}
        \begin{subsection}{Register substitute teacher)
        A class has to have a single "main" teacher. But it could happen that a single class is managed by multiple teachers. Or that, in case of the unavailability of the main teachers to class can be managed by the other subsitute teacher.
        \end{subsection}
        \begin{subsection}{Replace teacher in case of the unavailability of a teacher}
        In case of the long term unavailability of a "main" teacher, it should be possible to register another main teacher to that class (replace with that class his current subsiute), this has to be done on the organization-level. 
        \end{subsection}
        \begin{subsubsection}{Get restricted question set}
            Since the teacher can open/close a competition he must also have access to the restricted question sets. This is done by going
            to the 'competition management' as described in the section \nameref{sec:organizer}. On this page he can view all sets of questions.
        \end{subsubsection}
        \begin{subsubsection}{Get marks}
            After the competition has been closed, the teacher can view the marks of the pupils which are saved on the server.\\
            This can be done by opening a competition and using the 'Get Marks' button which opens the specific file.
        \end{subsubsection}
        \begin{subsubsection}{Get state of pupil}
            The teacher can monitor his pupils while joining a competition. This is done on the overview screen of 		the current running competition. On this page a 'View stats' button is available which displays a live table 
            (i.e. a table which is dynamically updated during the running competition). This table shows information such as which pupils started the competition, if they had already pressed the finish button or the time left. 
        \end{subsubsection}
        \begin{subsubsection}{Grant pupil minutes}
            On the page of the running competition, the teacher can also see a table with all his pupils that are competing. For each pupil he can see the 'time left' in the competition. It is also 		possible for the teacher to add/remove some minutes of a pupil. This is done by clicking on an arrow button which increases/decreases the 'time left' value.
        \end{subsubsection}
        \begin{subsubsection}{Moving ongoing competition status to different computer}
            When a student is participating in a competition and the network connection fails, the teacher has the possibility to move the status for that single instance of that competition to a different computer. On the computer with the failing network connection the teacher can click on "access the competition-instance ID" to get an competition-instance ID that contains all the answers of that student and can use this token in combination with the students ID to promote a new single instance of the ongoing competition to the same status and the competition-instance ID describes. This should happen on the teachers account. The way that the teachers does is by clicking on the students new session in the contest dashboard and promoting that session with the competition-instance ID. The student should get an update that his new session is updated. 
        \end{subsubsection}

        \begin{subsubsection}{Merging two students}
            When a teacher accidently creates a second account for a student
            who'd already got an account, he can merge the two accounts. This
            will merge the class history, current class and statistics of the
            two users. The teacher can specify which user is the persisting one,
            and in case the personal information and current class clash, this
            users data will be picked.
        \end{subsubsection}
        \begin{subsection}{Dry-run competition}
        A teacher should be able to do a dry-run of unrestricted competitions. A dry-run here means that he should be able to run through the competition without saving any information about the competition session. The purpose of this is so that  a teacher can demostrate a competition to his students or that a teacher can test if everything is working the way it should.
        \end{subsection}
    \end{subsection}

\end{section}
