
\begin{section}{Teachers}

    This section provides an overview of all functionality a teacher is offered
    on our application.

    \begin{subsection}{Inheritance}

        The teacher can do many things also available for other users. These are
        listed below. More explanation can be found with the respective users
        they share these functionality with.

        \begin{subsubsection}{Anonymous User}

            \begin{itemize}
                \item Do unrestricted quizzes;
                \item Log in.
            \end{itemize}

        \end{subsubsection}

        \begin{subsubsection}{Independent}

            \begin{itemize}
                \item Change password           (\ref{indep-pass});
                \item Change email address     (\ref{indep-email});
                \item Change preferred language (\ref{indep-lang});
                \item Log out                   (\ref{indep-logout}).
            \end{itemize}

        \end{subsubsection}

    \end{subsection}

    \begin{subsection}{General functionality}
	The teacher will use alot of specifications given by the organizer (like managing personal information\\
	, pupils, classes etc.). In general a teacher can do the following actions.\\
	\begin{subsubsection}{Register at organisation}
	As seen in the organizers section we will use a specific teacher management page. This will be used\\
	to register teachers.\\
	The teacher must inform the organizer by sending an email. Afterwards the teacher will have the\\
	possibility to enter the following information. This is done with the use of an webpage which has been send\\
	to the teacher by the organizer.
	\begin{itemize}
		\item Add account information (id and password and phonenumber)
		\item Add phonenumber
		\item Add school(s)
	\end{itemize}
	\end{subsubsection}
	\begin{subsubsection}{Change password}
	The teacher visits a special webpage managed by the organizer where it is possible to edit personal information.
	\end{subsubsection}
    \begin{subsubsection}{Change personal information}
	This is done as described above.
	\end{subsubsection}
	\begin{subsubsection}{Register pupil, class}
	This will be done as described in the organizers use-case. The teacher visits the ClassGroup managment page\\
	which is used to register pupils or classes and also to assign pupils to classes.
	\end{subsubsection}
	\begin{subsubsection}{Open and close competition + open and close local competition}
	Also this is very similar as in the organizers section. The teacher will use the organizer's competition page.\\
	On this page the teacher can add simply classes which he has registered as described in section Register pupil, class.\\
	A class must be added in order to participate in the official Bebras competition. After adding a class the teacher can\\
	close the competition. 
	\end{subsubsection}
	\begin{subsubsection}{Get restricted question set}
	\end{subsubsection}
	\begin{subsubsection}{Get marks}
	After the competition has closed, the teacher can view the marks of the pupils which are saved on the server.\\
	This could be done by opening a competition and using the 'Get Marks' button which opens the specific file.
	\end{subsubsection}
	\begin{subsubsection}{Get state of pupil}
	As described in the srs the teacher can monitors his pupils while joining a competition. This is done by a dashboard\\
	and a 'View stats' button which displays a live table (i.e. a table which is dynamically updated during the running competition.\\
	This table shows information like if they are logged in, if they had already pressed the finish button or the time left.
	\end{subsubsection}
	\begin{subsubsection}{Grant pupil minutes}
	This is an extra option which is presented by the dashboard of the teacher. 
    \end{subsubsection}
    \end{subsection}

\end{section}
