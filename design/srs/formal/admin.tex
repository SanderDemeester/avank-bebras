\begin{section}{Admin}

When the admin is successfully logged in, he will be redirected to his dashboard. The administrator has the default user functionalities as described in the relevant sections. You can read these for more information. Furthermore, the admin will have some extra links presented that are exclusively available to administrators. \\
The admin is responsible for the activation and deactivation of user accounts and he's also able to reset passwords for everyone. For this, he will have some extra features on the appropriate pages.

The administrator is all powerful in general, he's at the top of the hierarchy. Thus, he will have the same rights as the organizer.
\begin{itemize}
\item Manage Pupils
\item Manage Teachers
\item Manage Classgroups
\item Manage Competitions
\item Manage Question sets
\end{itemize}

\subsection{Manage organizers}
Besides this, he can also manage organizers. This page is similar to the other pages for management. On this page, there will be a management table (like the default one mentioned before) and a form to add an organizer. Here he can see information about all the organizers who are registered. This information contains:
\begin{itemize}
\item Bebras-ID
\item Name
\item Email address
\end{itemize}

There will also be some buttons for editing:
\begin{itemize}
\item Delete
\item Edit
\item Mimic
\end{itemize}

With these functions, the admin can delete someone as organizer and edit the information about an organizer. While mimicing an organizer, he will be able to do anything that organizer can do, but nothing more. During this, the admin loses his exclusive rights until he stops and returns to his account.

\subsection{Age levels}
On this page, the admin can add levels or edit existing levels. There will be a list of existing levels and beside of every level an edit button to change the name of the level. At the bottom of the list, there is an add button to add new levels to the list. An example can be found in part \ref{levels}. To define an age level, the admin must specify the minimum and maximum age. 

\subsection{Difficulty levels}
As with age levels, the admin can also create and manage difficulty levels. To define a difficulty level, the admin must enter a description of how hard or how easy a level is.

\subsection{Activation and deactivation of user accounts}
Just as described how organizers can \nameref{sec:organizer_block_pupils}, the administrator can do the same and even more, he can also block teachers and organizers.

\subsection{Add homepage links}
On the dashboard of the admin, he will have a link to change the homepage. Here, he can add links to the homepage by filling a URL in a textfield and clicking an add button. 

\subsection{Reset passwords for everyone}
On the personal page of every user, there will be a button visible for the admin to reset the password. When clicked, the user will get a message that his password is reset.

\end{section}
