\begin{section}{Admin}

When the admin is successfully logged in, he will be redirected to his dashboard. The administrator has the default user functionalities as described in the relevant sections. You can read these for more information. Furthermore, the admin will have some extra links presented that are exclusively available to administrators. 
\\

The administrator is all powerful in general, he's at the top of the hierarchy. Thus, he will have the same rights as the organizer.
\begin{itemize}
\item Manage Pupils
\item Manage Teachers
\item Manage Classgroups
\item ...
\end{itemize}

Besides this, he can also manage organizers. This page is similar to the other pages for management. On this page, there will be a management table (like the default one mentioned before) and a form to add an organizer. Here he can see information about all the organizers who are registered. This information contains:
\begin{itemize}
\item Bebras-ID
\item Name
\item Email address
\end{itemize}

There will also be some buttons for editing:
\begin{itemize}
\item Delete
\item Edit
\item Mimic
\end{itemize}

With these functions, the admin can delete someone as organizer and edit the information about an organizer. While mimicing an organizer, he will be able to do anything that organizer can do, but nothing more. During this, the admin loses his exclusive rights until he stops and returns to his account.

The admin is responsible for the activation and deactivation of user accounts and he's also able to reset passwords for everyone. For this, he will have some extra features on the appropriate pages.

\end{section}
