\section{The competition}

We shall first describe how an anonymous quiz is perceived by the user. Afterwards we shall explain 
the differences with restricted and official contests.

An anonymous user can visit a page with a list of quizzes where he is able to choose a quiz corresponding 
with one of the four categories (Ewok, Wookie, Padawan and Jedi). After he selects a quiz, he is directed to a web page with a `start' button. Pressing that button 
brings him to the competition and makes the clock tick. (During the competition, the time that remains
is displayed on the screen.) 

When the competition starts, the user is presented with the first question. Answering a question brings
him automatically to the next question. Moreover, at any time he can skip a question or return to an earlier 
one. Therefore the list with questions must be visible during the contest, this makes it easy for the user
to navigate between the questions. The user can also change his answers or even wipe answers out (because 
wrong answers cost marks while blank answers do not). In the list of questions, each question has some 
sort of indication whether the question is answered or not. This makes it easy for the user to have an
overview over the question set. To be able to know the importance of each question, the user can, for each
question, see how many points he can lose or win. 

On each question page the difficulty of that question is shown, together with a symbol that identifies
regular or official questions.

There is a button which the user can click on to indicate that he is finished. When the user presses this 
button, a warning will appear with the message that the user still has some time left to review his
answers. The user can then decide to permanently stop or to review his answers. The contest also ends
automatically when the time limit expires. In both cases the user then receives his marks and can review his
answers and read the corresponding feedback pages. If the user wants to review his marks and answers some 
other time, he will have to be registered as an Independent User. 

The official contest runs in the same way, except that the results (and feedback) are only available
when the competition is officialy closed (by the organizers). Moreover, the competition can only be taken
when it is officially opened by the organizers and by the teacher: the teacher must indicate which classes
of the schools for which he is responsible are allowed to take part in the official competition. An official
contest is organized by the different countries. The question set for this contest, consists of two parts. 
A part chosen by the international committee and a national selection of additional questions. On national 
scale, it is also possible to include a second round. 

For the restricted contests the situation is a little more complicated: in this case the teachers are fully 
in charge. A teacher may decide to hold a local competition. He starts by creating a new competition using 
the question sets of contests the organizers have opened to the teachers for this purpose. The contest 
should have 1 question set for each level that will participate. After the restricted contest is created, 
the teacher can allow any number of his classes to participate in it. Afterward, statistics will be 
available for each class individually and for all classes combined. It is possible that two or more teachers 
want to create a contest together. In order to do this, one of them will have to be the originator. He will 
be in charge of creating the contest as usual, and then inviting the other teachers (using their Bebras ID) 
to this contest. Each invited teacher can then also allow any of his classes to participate. Before any 
pupils can start the contest, the originator has to explicitly open it. This does not have to happen at the 
same time for each class in the competition. As with the official competition, marks and feedback will only 
be available after the competition is closed, this time by the originator.

For both the restricted and official competitions we also store the total time taken by the pupil. It can 
therefore be important not to wait too long to press the `finished' button.     

When the competition is in progress, the question files will be served from different server and will be proxyed through the main bebras webserver. The purpose of this is that a user of the system will not be able to see where the question are coming from. 