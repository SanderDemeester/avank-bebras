\section{In case of trouble}

We all know that schools do not always have the most reliable Internet connection and for that matter also not the most reliable PCs. What to do in case of failure?

If the Internet fails for a long time during the official competition, then not much can be done. However, the browser will store some information on the answers already given in a cookie on each pupil PC. The value of that cookie can be retrieved by the teacher and then later be sent to the application to serve as a basis for (at least partial) marking. If the internet only fails for a short period, the teacher can obtain a string-based representation of the information on the answers already given, by pressing a button on the pupil's computer. The teacher will then be able to insert this string into the application and the pupil can continue the contest with an optional amount of grace time. 

If a PC fails, then a pupil can move to another PC. Normally the same user is not allowed to take the same competition twice, but in this case the teacher can `reset' a pupil account in such a way that he only needs to log in on the new PC to continue the competition where he left off. And the pupil can even be given some extra `grace minutes' at the teacher discretion. 