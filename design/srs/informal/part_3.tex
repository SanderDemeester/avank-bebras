
\section{Classes and Schools}

A teacher is responsible for a number of \emph{classes}. A class belongs to a \emph{single} school, which implies that a school consists of multiple classes. Teachers may be responsible for classes in different schools. Classes have a number of registered pupils. Teachers assign pupils to classes, so when a pupil registers himself on the site, later a teacher will add him to a class, the pupil will get some sort of notification. At any time a pupil can belong to at most one class. 

Organizers register teachers with schools, each teacher can acces all students in the schools where they teach and they can create classes for their students.
When a teacher get removed from a school (and also his class) the class is deleted and needs to be recreated by the substituted teacher.

Classes must be recreated every school year; each class has an expiration date on this. Inside the application there is no connection between class "5ecoB" of 2013-2014 and "5ecoB" of 2014-2015. Most pupils will have moved to a different class, classes can change teachers and teachers can reassign students to different classes, and in fact, a class may disappear altogether (and reappear a few years later).

In particular this means that at the start of the next school year, every pupil needs to be registered anew by this teacher (whom might be different from last year, or even from a different school). A student can (and preferably should) then be registered with the same Bebras id as before. Pupils keep access to their history: how much marks they obtained for which contest in which year. 
When a teachers assigns a student to a class, the student can take part in official competitions.
(Statistics of past contests should also not be removed from the system).

Each teacher has on his page a list of his previous classes so he can access information about that class, and on that same page he has a list of all his currently active classes. 
The class to which a pupil belongs, determines the \emph{category} of that pupil. We support four different levels: \label{levels}
\begin{itemize}
\item Age 10-12, last two years of primary school. (Ewok)
\item Age 12-14, first two years of secondary school. (Wookie)
\item Age 14-16, third and fourth year of secondary school. (Padawan)
\item Age 16-18, fifth, sixth (and possibly seventh) year of secondary school. (Jedi)
\end{itemize}

