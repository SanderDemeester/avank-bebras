
\section*{2 Roles and ids}

There are seven kinds of users: Anonymous Users, Independent Users, Pupils,
Teachers, Organizers, Authors and Administrators.

\textit{Anonymous Users} are visitors to the website. They haven't logged in or
identified in any other way. This means they won't have any personal
statistics. Their results are however included in ``general'' statistics, as
how many users succeeded which contest.

When one of these anonymous users wants to keep track of his results, he can
register as a \textit{Independent User}. The user is now able to login. We can
keep statistics about him, which he'll be able to view on his profile page. He
can change his personal information, which include but are not limited to his
name, age and email-address.

A \textit{Pupil} belongs to a certain class in a certain school, registered
as such by his teacher. He can take part in all competitions (restricted and
official) that were opened for his class by his teacher. He also can
participate in unrestricted contests. When the teacher disbands the class at
the end of the year, a pupil is degraded to an independent user. He will
retain the results of his restricted and official contests, but will not be
able to participate until his next teacher puts him in a class.

A \textit{Teacher} is associated with one or more classes (possibly at
different schools). He can invite independent users for his classes, or create
new accounts himself. With the restricted question sets, he can organize
training competitions for his pupils. He can also supervise the official
competition for his pupils during the official competition week. A teacher
must register by sending an e-mail request to the organizers.

An \textit{Organizer}, of which there are only a few, can upgrade independent
accounts to teachers, or just create whole new accounts. They can use the
application to upload new questions and complete sets. With the uploaded
questions they can create and expand question sets, or just delete them.
Organizers can mimic teachers: they can instruct the system to behave as if a
certain teacher was logged in. After they log out, they then automatically
resume their role as organizer.

Furthermore, there are Question \textit{Authors}. They write the questions
as HTML pages, which are specified below. Question authors cannot directly
add the questions to the application. Their questions are first reviewed and
then rejected or accepted by an organizer. Often, organizers will be authors
themselves. Question authors do not get a special account for the application.

Last but certainly not least, we have the \textit{Administrator}. He can create
or modify organizer accounts, and about everything else. The little things he
cannot do, he mimics through the accounts of organizers, teachers, pupils and
independents. They can imitate one of these accounts in their own name, or
log in as another user.

Except for the anonymous users and the question authors, everyone is identified
with a user id (\textit{Bebras id}) and a password. When the account is
created, an initial password is assigned, which should be changed by the user.
Of course, when a user registers himself, he can choose his own password.
Teachers can reset the passwords for their pupils, and organizers can do the
same for teachers. Administrators can reset passwords for everyone. User
accounts can be (temporarily) deactivated, in the same hierarchy.
