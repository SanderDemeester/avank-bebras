
This document describes a web application for the Belgium version of the
international Bebras contest. It allows (classes of) primary and secondary
school pupils to take part in the (international) Bebras contest over the
Internet.

The Bebras contest (also called `beaver competition', `beverwedstrijd', ...) is
an international online competition for young people of age 10 up to 18 that
aims to promote computer science and computer fluency among children and young
pupils. Every year an international team puts together a series of questions
which can then be used by the individual countries to set up a contest. (Each
country selects and translates the questions, and they are also allowed to add
some of their own.)

The Beaver contest usually takes place around mid-November and is open for a
complete week. Its practical organization is done through schools and their
teachers: pupils must register with their teachers and the competition is run
from the PC rooms in the schools and not from the pupils' personal home
computers.

Although the official competition only lasts a single week, most countries also
allow earlier competitions to be tried out online, often without the need for
prior registration.

More information about the Bebras contest can be found on the web:
\begin{itemize}
    \item \texttt{www.bebras.org}: Official international web site.
    \item \texttt{www.bebras.be}: Web site of the Belgian chapter of the competition.
    \item \texttt{informaticaolympiade.nl}: Hosts the Dutch version of the competition.
    \item \texttt{www.informatik-biber.de}: Bebras in Germany.
    \item \texttt{www.informatik-biber.ch}: Bebras in Switzerland.
\end{itemize}

