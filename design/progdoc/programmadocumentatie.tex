\documentclass[]{article}
\usepackage[a4paper]{geometry}
\usepackage[dutch]{babel}
\usepackage{listings}

\setlength{\parindent}{0pt}
\setlength{\parskip}{14pt}

\title{
    Programmadocumentatie
}

\begin{document}

\maketitle

\section{Registreren}
Om te registreren en een Bebras account aan te maken klik je op 'Sign up'. Hierdoor kom je op de pagina om een account aan te maken. [TODO: screenshot sign-up page] Hier vul je de gevraagde velden in zoals daar zijn: naam, e-mailadres en geboortedatum. Je duidt je geslacht aan en kiest de taal die je voorkeur wegdraagt. Onderaan vul je een zelfgekozen wachtwoord in. Dit herhaal je in het veld eronder. Klik dan op 'Maak account'. Je krijgt een Bebras-ID toebedeeld en kan hiermee inloggen.

Indien een verplicht veld niet is ingevuld, zal je hier een melding van krijgen. 

\section{Inloggen}
Nu je een Bebras account hebt aangemaakt, kun je ook inloggen. Hiervoor klik je op de startpagina op 'Sign in'. [TODO: screenshot startpagina] Vul je Bebras-ID en wachtwoord in, klik op 'login' en je bent ingelogd! Als alles goed gaat, zie je nu jouw persoonlijke pagina. In het geval dat je een foute gebruikersnaam of wachtwoord hebt ingevuld, zal je hiervan een melding krijgen.

\section{Wachtwoord vergeten}
In het geval dat je je wachtwoord niet meer weet, kan je klikken op 'Wachtwoord vergeten' dat zich op het aanmeldvenster bevindt naast de loginknop. Hier vul je jouw Bebras-ID in en het e-mailadres waar een nieuw wachtwoord naar gestuurd zal worden. Indien je zelf geen e-mailadres zou hebben, vul je hier het e-mailadres van jouw leerkracht in.

\section{Independent}

Nu je aangemeld bent, beschik je over een aantal standaardfuncties.

\subsection{Profiel bewerken}
Om je profiel te bewerken, klik je rechtsboven op je Bebras-ID zodat er een menu wordt geopend. [TODO: screenshot account-dropdown] In dit menu kies je voor 'Instellingen'. [TODO: screenshot personaledit] Hier kan je de verschillende gegevens die op jouw profiel verschijnen aanpassen. Om de aanpassingen op te slaan, klik je op 'Opslaan'.

\subsection{Wachtwoord veranderen}
Om je wachtwoord te veranderen, vul je jouw huidig wachtwoord in. Geef dan je nieuwe wachtwoord in en bevestig dit in het veld eronder. Klik op 'Ok' om je wachtwoord te wijzigen.

\subsection{Uitloggen}
Om de webapplicatie te verlaten, klik je rechtsbovenaan op je gebruikersnaam. Er zal een menu verschijnen waar je onderaan voor 'Sign out' kunt kiezen.

\subsection{Vrageneditor}

Om een eigen vraag toe te voegen ga je naar de 'Vraagstukontwerper'. [TODO: screenshot questioneditor] Bij het maken van een vraag, heb je verschillende keuzes. Of je maakt een meerkeuzevraag, of een open vraag. Voor meer ervaren gebruikers is er ook de mogelijkheid een vraag te uploaden in de vorm van een ZIP-bestand.

\subsubsection{Meerkeuzevraag}

[TODO: screenshot MCQ] Om een meerkeuzevraag aan te maken, kies je eerst de taal waarin de vraag wordt opgesteld. Klik dan op de plus knop om een vraag in deze taal toe te voegen. Nu zie je 4 tabbladen verschijnen, namelijk: 'Algemeen', 'Antwoorden', 'Vraag' en 'Evaluatie'. Bij het tabblad 'Algemeen' vul je de titel van de vraag in. Onder 'Antwoorden' is het de bedoeling dat je de 4 mogelijke antwoorden geeft en aanvinkt welke van de 4 de correcte oplossing is. Bij het tabblad 'Vraag' kun je de vraag ingeven in het hiervoor voorziene tekstvak. Boven het tekstvak vind je enkele knoppen om de opmaak te verzorgen. Het tabblad 'Evaluatie' lijkt sterk op het vorige. Hier is het de bedoeling dat je de feedback ingeeft die de participant te zien krijgt nadat hij deze vraag beantwoord heeft in een wedstrijd. 

Eens alles ingevuld is, klik je op 'Verzend'. Indien er zich geen foutmeldingen voordoen, is de vraag succesvol ingediend. In het andere geval zal je gevraagd worden om alles nog eens na te lezen en de hiaten in te vullen.

Als je dezelfde vraag in een andere taal wil toevoegen, kies je die taal en voeg je deze toe door dezelfde procedure te volgen. Wil je de vraag in een bepaalde taal verwijderen, klik dan op het kruisje dat naast de taal staat en bevestig je actie. 

Voorts heeft men ook nog de mogelijkheid om de opgestelde vraag in de verschillende talen te downloaden in de vorm van een ZIP-bestand. 

\subsubsection{Open vraag}

Bij het maken van een open vraag, kan men dezelfde procedure volgen als bij het maken van een meerkeuzevraag. Het enige verschil is dat men onder 'Antwoord' geen 4 keuzemogelijkheden ingeeft, maar 1 correct antwoord.

\subsubsection{Importeren}
Om een vraag te uploaden, klik je op 'Importeer van .ZIP'. Er zal een dialoogvenster verschijnen waar men het gewenste bestand kan zoeken en uploaden. Dit ZIP-bestand moet aan bepaalde eisen voldoen. Zo zal er per taal een indexpagina en een pagina voor feedback aanwezig moeten zijn. Vanuit deze HTML-bestanden kan verwezen worden naar andere bestanden, zoals afbeeldingen, die in het ZIP-bestand kunnen meegeleverd worden. Daarenboven moet er zich ook een XML-gebaseerd configuratiebestand in het ZIP-bestand bevinden. 

Dit XML-bestand, dat altijd de naam "`question.xml"' moet hebben, bevat
\begin{itemize}
\item de beschikbare talen
\item de startpagina van de vraag per taal
\item de titel van de vraag voor alle beschikbare talen
\end{itemize}

Voorts wordt er onderscheid gemaakt tussen meerkeuzevragen en open vragen.

\textbf{Structuur:}

Het XML-bestand bestaat uit een wortelknoop \verb+<root xmlns="bebras:Question">+ 
die op zijn beurt de soort vraag herbergt: \verb+<multiple-choice-question>+ of \verb+<regex-question>+.
Het 'vraag'-element bevat per beschikbare taal (\verb+<language code="xx">+) 
de start- en feedbackpagina (\verb+<index>index_xx.html</index>+ en \verb+<feedback>feedback_xx.html</feedback>+), 
de titel van de vraag tussen \verb+<title>+ tags 
en het antwoord (\verb+<input regex=""/>+) / de mogelijke antwoorden (\verb+<answers><answer></answer></answers>+)
met een indicatie bij het correcte antwoord door de optie \verb+correct="true"+ mee te geven.

Voorbeelden:
\lstinputlisting[language=XML, frame=single]{xml/MCQ.xml}
\lstinputlisting[language=XML, frame=single]{xml/open_question.xml} 

\section{Administrator}

\subsection{Beheer Veelgestelde vragen}

\subsubsection{Bekijk veelgestelde vragen}
Deze pagina geeft een overzicht van alle veelgestelde vragen.

\subsubsection{Bewerk veelgestelde vragen}
Op deze pagina zie je de lijst van alle veelgestelde vragen. Door op een kolomnaam te klikken, wordt de tabel automatisch op deze kolom gesorteerd. Om de volgorde om te draaien, klik je nogmaals op diezelfde kolomnaam. Door een taal in tegen, kan men de lijst filteren en enkel de vragen in die taal over houden. Onderaan de tabel kan men bladeren doorheen de lijst in het geval er meerdere pagina's met vragen zijn. Om een vraag te bewerken, klik je op de 'Bewerk'-knop. Een vraag verwijderen uit de lijst kan door op het kruisje te klikken.

\subsubsection{Maak een nieuwe veelgestelde vraag aan}
Om een nieuwe veelgestelde vraag toe te voegen, klik je op 'Voeg nieuwe FAQ toe'. [TODO: screenshot newFAQ] Op deze pagina vul je de vraag en antwoord in in de daarvoor bestemde (en vereiste) velden. Daarna kies je de taal waarin de vraag is opgesteld en klik je op 'Save' om de vraag op te slaan. Je hebt ook de mogelijkheid om de vraag niet op te slaan en de actie te annuleren.

\subsection{Serverbeheer}

\subsubsection{Bekijk servers}
Deze pagina geeft een overzicht van de servers die momenteel gebruikt worden. Bovenaan kan men filteren op naam alsook een nieuwe server toevoegen. [TODO: screenshot server]

\subsubsection{Voeg server toe}
Om een nieuwe server toe te voegen, vul je de gevraagde velden in en klik je op 'Maak aan'. Om de actie te annuleren, klik je op 'Annuleer'.

\subsection{Vragenbeheer}

\subsubsection{Bekijk vragen}
Deze pagina geeft een overzicht van de servers die momenteel gebruikt worden. Bovenaan kan men filteren op ID. [TODO: screenshot questions]

\subsubsection{Goedkeuren van vragen}
Hier krijg je een overzicht van alle ingediende vragen met vermelding van de auteur en de indiendatum. [TODO: screenshot ingediende vragen] Hier kan de administrator een vraag goedkeuren of afkeuren door op de desbetreffende knoppen te klikken. Bij het goedkeuren van een vraag, kent de administrator een officieel ID toe aan de vraag zodat deze later kan teruggevonden worden. Ook kiest hij een server waarop de vraag geplaatst wordt en bepaalt hij of de vraag momenteel actief is. Eens de vraag goedgekeurd is, wordt ze verwijderd uit de voorlopige lijst en toegevoegd aan de officiële lijst van vragen.

\end{document}