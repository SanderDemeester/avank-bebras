\documentclass[]{article}
\usepackage[a4paper]{geometry}
\usepackage[dutch]{babel}

\setlength{\parindent}{0pt}
\setlength{\parskip}{14pt}

\title{
    Programmadocumentatie
}

\begin{document}

\maketitle

\section*{Registreren}
Om te registreren en een Bebras account aan te maken klik je op 'Sign up'. Hierdoor kom je op de pagina om een account aan te maken. [TODO: screenshot sign-up page] Hier vul je de gevraagde velden in zoals daar zijn: voornaam, familienaam, e-mailadres en geboortedatum. Je kiest de taal die je voorkeur wegdraagt en duidt je geslacht aan. Onderaan vul je een zelfgekozen wachtwoord in. Dit herhaal je in het veld eronder. Klik dan op 'Maak account'. Je krijgt een Bebras-ID toebedeeld en kan hiermee inloggen.

\section*{Inloggen}
Nu je een Bebras account hebt aangemaakt, kun je ook inloggen. Hiervoor klik je op de startpagina op 'Sign in'. [TODO: screenshot startpagina] Vul je gebruikersnaam en wachtwoord in, klik op 'login' en je bent ingelogd! Als alles goed gaat, zie je nu jouw persoonlijke pagina.

\section*{Vraag aanmaken}
Wanneer je op de 'question editor' bent gebotst, was dat zeker met de intentie om een vraag aan te maken. Hier zit je alvast op de goede plek. [TODO: screenshot questioneditor] Bij het maken van een vraag, heb je verschillende keuzes. Of je maakt een meerkeuzevraag, of een open vraag, of je importeert onmiddellijk een volledig arsenaal van vragen. 

[TODO: screenshot MCQ] Om een meerkeuzevraag aan te maken, kies je eerst de taal waarin de vraag wordt gesteld. Klik dan op de plus knop om een vraag in deze taal toe te voegen. Nu zie je 4 tabbladen verschijnen, namelijk: 'Algemeen', 'Antwoorden', 'Vraag' en 'Feedback'. Bij het tabblad 'Algemeen' vul je de titel van de vraag in. Onder 'Antwoorden' is het de bedoeling dat je de 4 mogelijke antwoorden geeft en aanvinkt welke van de 4 de correcte oplossing is. Bij het tabblad 'Vraag' kun je de vraag ingeven in het hiervoor voorziene tekstvak. Boven het tekstvak vind je enkele knoppen om de opmaak te verzorgen. Het tabblad 'Feedback' lijkt sterk op het vorige. Hier is het de bedoeling dat je de feedback ingeeft die de participant te zien krijgt nadat hij deze vraag beantwoord heeft in een wedstrijd. 

Eens alles ingevuld is, klik je op 'Submit'. Indien er zich geen foutmeldingen voordoen, is de vraag succesvol ingediend. In het andere geval zal je gevraagd worden om alles nog eens na te lezen en de hiaten in te vullen.

Als je dezelfde vraag in een andere taal wil toevoegen, kies je die taal en voeg je deze toe door dezelfde procedure te volgen. Wil je de vraag in een bepaalde taal verwijderen, klik dan op het kruisje dat naast de taal staat en bevestig je actie. 

Voorts heeft men ook nog de mogelijkheid om de opgestelde vraag in de verschillende talen te downloaden in de vorm van een zip-bestand. 

Bij het maken van een open vraag, kan men dezelfde procedure volgen als bij het maken van een meerkeuzevraag. Het enige verschil is dat men onder 'Antwoord' geen 4 keuzemogelijkheden ingeeft, maar 1 correct antwoord.

Om een zip-bestand van vragen te uploaden, klik je op 'Importeer van .ZIP'. Er zal een dialoogvenster verschijnen waar men het gewenste bestand kan zoeken en uploaden.
\end{document}