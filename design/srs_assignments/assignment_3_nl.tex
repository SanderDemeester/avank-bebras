
\begin{section}*{Opdracht 3}

Als we willen voorkomen dat een gebruiker 2 (of meerdere keren) ingelogd is, dan kunnen we dit doen op 2 manieren. Als een gebruiker in wil loggen terwijl hij al ergens anders ingelogd is, dan kunnen we dit gewoonweg niet toe laten. Dit levert echter een probleem op als de gebruiker vergeten uit te loggen is en de sessie gesloten heeft. Dan kan hij namelijk totaal niet meer inloggen. Een tweede oplossing is om de eerste login ongedaan te maken. Dit is de optie waarvoor wij zullen kiezen. Merk op dat "inloggen" hier zowel echt inloggen als nabootsen van een gebruiker kan zijn.
\\\\
Technisch zullen we het als volgt aanpakken. Om het inloggen te laten werken moeten we op een of andere manier een Session ID verbinden met een Bebras ID. Wij zullen dit doen m.b.v. een hashmap die Strings (Session ID) verbindt met een een stapel van Bebras ID. Deze stapel is om het nabootsen toe te laten. Als een leraar een leerling nabootst, dan wordt de leerling zijn ID boven de leraar zijn ID gepusht. Als de leraar dan uit de leerling zijn account uitlogt, dan is hij terug op zijn account.
\\\\
Als een gebruiker wil 'normaal' inloggen, dan overlopen we eerst de lijst van ingelogde gebruikers (de volledige stapels). Als het Bebras ID er al tussenstaat, dan zijn er twee opties. Ofwel is dit een 'normale' login, en dan wordt de oudere sessie gewoon verwijdert. De andere optie is dat dit een nabootsing is, m.a.w. iemand die hoger op rechten-rangorde staat is ingelogd op die gebruiker zijn account. In dit geval zal de gebruiker niet kunnen inloggen tot die andere gebruiker uit zijn account is uitgelogd.
\\\\
Als een gebruiker een andere wil nabootsen, dan wordt ook weeral de volledige lijst overlopen. Als het ID er al tussenstaat zijn er weeral 2 opties. Dit kan een 'normale' login zijn, en in dit geval wordt een boodschap getoont aan de gebruiker die wil nabootsen. Deze boodschap geeft wat informatie over wat de ander gebruiker aan het doen is. Aan de hand van deze informatie kan de gebruiker beslissen of hij wil nabootsen of niet. Als hij dit doet, dan wordt de andere sessie volledig verwijderd. Als deze andere gebruiker op dat moment zelf iemand aan het nabootsen was, dan wordt hij ook daar uitgelogd. De overige optie is weeral dat deze login een nabootsing is. Hier zijn weer twee opties. Ofwel is het iemand met dezelfde rechten (bvb. een Organizer wil een Teacher nabootsen die al nagebootst is door een andere Organizer). In dit geval gebeurt hetzelfde als ware het een normale login. Indien de Organizer doorgaat met nabootsen wordt de andere gebruiker enkel maar uit de nagebootste login verwijderd, en niet uit zijn normale login. Het laatste geval is dat de andere nabootsing gebeurt door iemand met hogere rechten. In dit geval zal de gebruiker de andere gebruiker niet kunnen nabootsen.
\\\\
Het is dus een stapel-gebaseerde login, waar steeds volledige toppen van verwijderd worden (startende bij de ID die dubbel zou ingelogd worden) waarbij iemand alleen maar iemand anders met dezelfde of lagere rechten kan kicken. Het is de bodem van de stapel die het hier relevante rechten-niveau aangeeft. Een Admin die een Organizer nabootst die op zijn beurt een Teacher nabootst heeft dus de rechten van een Admin, en dus hogere rechten dan een Organizer als beide bijvoorbeeld een Pupil willen nabootsen.
\end{section}