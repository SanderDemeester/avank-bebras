
\begin{section}*{Assignment 3}

If we want to prevent a user from logging in 2 (or more) times, then we have 2 options. The first option is to simply not allow the second login. This however poses a problem in cases where a user forgets to logout and closes the session. He is then still registered as 'logged in', but he can no longer log out. A second options is to undo the first login. This is the option we have chosen. Remark that in this context "logging in" can both mean really login in as mimicking a user.
\\\\
We will implement this solution as following. To be able to 'log in', we have to link a Session ID to a Bebras ID in one way or another. We will utilize a hashmap that links strings (Session ID) to a stack op Bebras IDs. We use a stack, as opposed to a single string, to be able to allow mimicking. For instance if a Teacher starts mimicking a Pupil, the pupil's ID will be pushed on top of the Teacher's ID. When the Teacher logs out of the Pupil's account he returns to his actual account.
\\\\
If a user(henceforth called 'the requester) wants to log in normally, we first iterate over the list of logged in users (we check the full stacks, not only the top). If the ID is already in there, then there are possibilities. The first possibility is that this is a 'normal' log in. In this case the requester is prompted a confirmation screen, which informs him that his account is already logged in. If he choses to continue with the login, the other session is terminated. The second possibility is that someone is mimicking his account, i.e. someone who is higher on the hierarchy is logged in under his account. In this case the requester is informed of this, and he can't log in until the other person logs out.
\\\\
If the requester wants to mimic another user we also iterate over the full list. If the ID is already in there there are 2 possibilities again. The ID can already be logged in normally, i.e. not mimicked. In this case the requester is shown a confirmation prompt. This prompt gives some info about what the other user is doing, and using the information the requester can decide if he wants to mimic. If he does, the other user is logged out. If that user happened to be mimicking another user at the same moment, then he is also logged out of that user's account. The other possibility is that the ID is being mimicked. Here there are 2 sub-possibilities. The user can be mimicked by someone who has equal or lower rights than the requester. In this case the requester is shown a similar prompt as the previous ones, and if he choses to continue, then the other user is logged out of the mimicked account. He is however not logged out of the accounts who are lower on the stack. The user can also be mimicked by a user with higher rights. In this case the requester is informed of this and he can't mimic the user until the other user logs out of the account.
\\\\
To summarize: it is a stack-based login, in which we always remove full heads (starting with the ID that would otherwise be logged in twice), in which someone can only kick someone with equal or lower rights. It is the bottom of the stack that determines the effective rights level. An Admin who mimics an Organizer who mimics a Teacher has the rights of an Admin, and as such has higher rights to mimic a pupil than an Organizer.
\end{section}