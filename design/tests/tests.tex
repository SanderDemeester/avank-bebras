\documentclass[10pt,a4paper]{article}
\usepackage[latin1]{inputenc}
\usepackage[dutch]{babel}
\usepackage{amsmath}
\usepackage{amsfonts}
\usepackage{amssymb}
\usepackage{hyperref}

\title{Test documentatie VOPRO}

\begin{document}

\section*{Testing: continuous integration}
Er wordt gebruikgemaakt van een continuous integration server die te vinden is op avank.ugent.be/jenkins.\\
Jenkins voert de play tests uit die door de verschillende leden zijn gemaakt. De locatie van deze tests is ''application/test''. \\

Aangezien het project gebruikmaakt van PostgreSQL als database moet deze tijdens het testen ook worden gesimuleerd. Daarvoor maken we gebruik van de "in memory"-database H2. Vermits onze data klassen geannoteerd zijn met @-notatie en we gebruikmaken van Ebean kan het Play Framework zelf de database relaties namaken in H2 en onze tests uitvoeren.\\

Om onze JUnit tests uit te voeren, is het uitvoeren van het commando "play test" voldoende.

\section*{Testing: regression testing}
Geautomatiseerde tests gebeuren door de OWASP ZAP Proxy. Het nut van deze proxy is het geautomatiseerd testen van de beveiliging van de webapplicatie. \\
Op de webserver loopt permanent een proxy instantie. Als laatste fase voert Jenkins het script ''application/scripts/security\_regressiontest.py'' uit. Dit script staget een versie van de webapplicatie (gebaseerd op dezelfde code die Jenkins gebruikt) en voert deze uit. Daarna zal de applicatie worden ondervraagd door een ZAP spider via de proxy. Het resultaat hiervan kan worden gevonden bij de Jenkins output. 

\section*{Java documentatie}
Tijdens het uitvoeren van de tests door Jenkins wordt ook de Javadoc ge"updatet. 
De laatste versie van onze documentatie is te vinden op avank.ugent.be/javadoc.

\section*{Database setup}
Wij gebruiken als databank postgresql. Als algemene authenticatie methode gebruiken we "md5", dus we laten postgresql zelf instaan voor het afhandelen van authenticatie. 
We maken gebruik van de databank "avank". De gebruiker "postgresql" met wachtwoord "sander" heeft daarop alle rechten (zoals is te zien in de folder application/conf/application.conf".
De play applicatie zelf praat met de databank via ebeans. De driver voor de communicatie is opgenomen als een play dependency en is te vinden in "Build.scala".\\

De locatie van de datbase scripts zijn te vinden op github:
\url{https://github.ugent.be/fvdrjeug/avank-bebras/blob/master/design/database/SQLscripts/createTablesComplete}
\href{https://github.ugent.be/fvdrjeug/avank-bebras/blob/master/design/database/SQLscripts/createTablesComplete}{database scripts}\\

Om deze uit te voeren maakt u verbinding met de databank met het commando
\begin{itemize}
\item psql -U postgres
\item $\backslash$ c
\item $\backslash$ i "locatie van het script".
\end{itemize} 
Let wel, hiervoor is het nodig dat "md5" de gebruikte authenticatie methode is.

\section*{System requirements}
Het is nodig dat er een sendmail daemon loopt op de server om emails te versturen.
\end{document}
